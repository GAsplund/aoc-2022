\documentclass{article}

\usepackage[margin=0.7in]{geometry}
\usepackage[parfill]{parskip}
\usepackage[utf8]{inputenc}

\usepackage{amsmath,amssymb,amsfonts,amsthm}

\usepackage{xstring}
\usepackage{catchfile}
\usepackage{stringstrings}
\usepackage{xifthen}
\usepackage[nomessages]{fp}
\usepackage{intcalc}

\begin{document}

    % Variables
    \newcommand{\filepath}{input.txt}
    \newtoks\screen
    \def\cycle{0}
    \def\xreg{1}
    \def\signalstrength{0}

    % CRT is 6 rows and 40 columns, aka 240 pixels

    \newcommand\checkcycle{
        % Part 1:
        \ifthenelse{\equal{\intcalcMod{\cycle}{40}}{20}}
        {
            \FPeval{\strength}{round(\cycle * \xreg, 0)}
            Cycle: \cycle\\
            Strength: \strength\\
            \FPeval{\signalstrength}{round(\signalstrength + (\cycle * \xreg), 0)}
        }
        {}

        % Part 2 draw sprite on screen: 
        \edef\curscreen{\the\screen}
        \FPeval{\cursor}{round(\intcalcMod{\cycle}{40}, 0)}
        \FPeval{\spriteRel}{round(\cursor - \xreg, 0)}

        % Add new line if screen has wrapped around
        \ifthenelse{\equal{\cursor}{1}}
        {
            \edef\newscreen{\curscreen%
            \\}
            \screen={\newscreen}
        }
        {}

        % Add lit or unlit pixel based on sprite position
        \ifthenelse{\equal{\spriteRel}{0} \OR \equal{\spriteRel}{1} \OR \equal{\spriteRel}{2}}
        {
            \edef\newscreen{\curscreen%
            $\blacksquare$}
            \screen={\newscreen}
        }
        {
            \edef\newscreen{\curscreen%
            $\square$}
            \screen={\newscreen}
        }
    }

    \textbf{Cycle checkpoints:}

    \newread\file
        \openin\file=\filepath
        \loop\unless\ifeof\file
            \read\file to\fileline % Reads a line of the file into \fileline
                \FPeval{\cycle}{round(\cycle + 1, 0)}
                \substring[q]{\fileline}{1}{4}
                \edef\instruction{\thestring}
                \ifthenelse{\equal{\instruction}{addx}}
                {
                    \checkcycle
                    \FPeval{\cycle}{round(\cycle + 1, 0)}
                    \checkcycle

                    % RIP highlighter, it thinks it's in math mode now.
                    \substring[q]{\fileline}{6}{$-1}
                    \FPeval{\xreg}{round(\xreg + \thestring, 0)}
                }
                {
                    \checkcycle
                }
        \repeat
    \closein\file

    % Finally, print solutions
    \textbf{Solution 1:} \signalstrength\\
    \textbf{Solution 2:}\\
    \the\screen

\end{document}
